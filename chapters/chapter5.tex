\chapter{Differentiation}

Short chapter here. The proofs are fairly straightforward and if you have some calculus background, this chapter should've have been too hard.
The hardest proof for me was L'Hospital's Theorem, namely the step where we have

\begin{equation*}
  \frac{f(x)}{g(x)} < r - r \frac{g(y)}{g(x)} + \frac{f(y)}{g(x)}
\end{equation*}

and we conclude by letting $x \to a$, that we have
\begin{equation*}
  \frac{f(x)}{g(x)} < q.
\end{equation*}
This is confusing because when $x \to a$, and we assume $g(x) \to \infty$ as $x \to a$, I don't see how the equation reduces nicely.
Meaning, it seems like $f(x)/g(x) \to 0$ as well, but maybe it's because we have that $y$ is fixed, so we know \textit{for sure} that
\begin{equation*}
  r \frac{g(y)}{g(x)} + \frac{f(y)}{g(x)}
\end{equation*}
all go to zero, so we are left with
\begin{equation*}
  \frac{f(x)}{g(x)} < r < q.
\end{equation*}

I've noticed the problems in this chapter tend to be pretty MVT (Mean Value Theorem) heavy, so make sure to pull that out when you can.

\section{Exercises}

\bx{
  I think we can change this into
  \begin{equation*}
    \frac{
      \abs{f(x) - f(y)}
    }{
      x - y
    } \leq x - y
  \end{equation*}
  and if we let $y \to x$, then we have
  \begin{equation*}
    \abs{f'(x)} \leq 0
  \end{equation*}
  which implies that $f'(x) = 0$, for all $x$, and thus $f$ is constant.
}

\bx{
  If we have $f'(x)>0, x \in (a, b)$, then it must be strictly increasing, because AFSOC $\exists x_1< x_2 \in (a, b), f(x_1) \geq f(x_2)$,
  then we can show that $\exists x_3 \in (x_1, x_2)$ such that
  \begin{equation*}
    f'(x_3) = \frac{f(x_2) - f(x_1)}{x_2-x_1} \leq 0
  \end{equation*}
  by the mean value theorem, which contradicts the fact that $f'(x) > 0 \forall x \in (a, b)$.

  Since $g(x) = f^{-1}(x)$, we have
  \begin{align*}
    g(f(x))       & = f^{-1}(f(x)) = x    \\
    g'(f(x))f'(x) & = 1\tag{product rule} \\
    g'(f(x))      & = \frac{1}{f'(x)}
  \end{align*}
  We only really needed $f'(x) \neq 0$ I believe, not the strictly increasing part.
}

\bx{
  For injective (one-to-one) proofs we usually start with $f(x_1) = f(x_2)$ and try to prove that $x_1 = x_2$.
  If we do that here, we get
  \begin{align*}
    f(x_1)                & = f(x_2)                \\
    x_1 + \epsilon g(x_1) & = x_2 + \epsilon g(x_2) \\
    g'(x_1)               & = g'(x_2)
  \end{align*}
  I'm honestly not sure what to do here, especially with the bounded derivative property of $g$.

  \TODO
}

\bx{
  The solution for this is to choose
  \begin{equation*}
    P(x) = \sum_{i=0}^n \frac{C_i}{i+1} x^{i+1}
  \end{equation*}
  and see that $P(0), P(1) = 0$ from the problem statement, and that $P'(0) = 0$ follows by the mean value theorem.

  What I can add to this problem is that I was confused about the
  \begin{equation*}
    P'(x) = C_0 + C_1x + \cdots + C_nx_n = 0
  \end{equation*}
  It seems like by taking $x = 0$, we can show that $C_0 = 0$. Now, if we differentiate to get $P''(x)$, we can take $x=0$ and show $C_1 = 0$ and so on...unsure why this is flawed\footnote{Maybe it's because I'm only considering $x=0$? At other $x\neq 0$ values, maybe these constants don't have these properties? I guess we're also not given that equation is true for all $x$, we just want to know for which $x$ we \textit{definitely} know it is true for.}.
}

\bx{
  Applying MVT to $f(x+1) - f(x)$ we get
  \begin{equation*}
    f(x+1) - f(x) = (x + 1 - x)f'(y)
  \end{equation*}
  for $y \in (x, x+1)$. Now, since we know that $\lim_{x \to \infty} f'(x) = 0$, we can choose some $x$ so large so that we have this choice of $y$ imply that
  $\abs{f'(y)} < \epsilon$, meaning that $g(x) = f(x + 1) - f(x) < \epsilon$, implying that $\lim_{x \to \infty} g(x) = 0$.
}

\bx{
  Here is a solution sketch to find motivation.

  If we want to show that $g$ is monotonically increasing, we need to show that $g'(x) \geq 0$.
  Given the definition of $g$, what we end up getting is that

  \begin{align*}
    g'(x) = \frac{f'(x)x - f(x)}{x^2} & \text{?}\geq 0
    x \cdot f'(x) \text{?}\geq f(x)
  \end{align*}

  Seeing we need a relationship between $f(x)$ and $f'(x)$, we can apply MVT to $f(x), f(0)$ and go from there.
  I had to look up the solution for this one, but I wanted to add in the motivation for the solution while I was trying to solve it.
}

\bx{
  This is pretty straightforward, we have
  \begin{equation*}
    \frac{f'(x)}{g'(x)}
    = \lim_{t \to x} \frac{f(t) - f(x)}{t - x}\frac{t-x}{g(t) - g(x)}
    = \lim_{t \to x} \frac{f(t) - f(x)}{g(t) - g(x)}
    = \lim_{t\to x}\frac{f(t)}{g(t)}
  \end{equation*}
  The last step comes from the fact that $f(x), g(x) = 0$.

  This holds for complex functions since we made no assumptions about $R^1$ operations here.
}

\bx{
  This is probably not formal, but since we know that $f'$ is continuous on $[a, b]$, we have that $f'(x)$ exists for $x \in [a, b]$, and that for small enough $\delta$, we can get
  \begin{equation*}
    \frac{f(t) - f(x)}{t - x}
    \label{eq:5.8}
  \end{equation*}
  arbitrarily close to $f'(x)$.

  The question then comes, why can't we do this when $f'$ \textit{isn't} continuous, and I think it's because without continuity, we can't show that the expression in \label{eq:5.8} can be arbitrarily close to $f'(x)$.

  Intuition says this holds for vector-valued functions since we have a finite number of coordinates and we just need to show the $\epsilon$ inequality holds for $n$ dimensions.
}